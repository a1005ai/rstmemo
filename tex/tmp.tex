\documentclass[a4paper]{article}
% generated by Docutils <http://docutils.sourceforge.net/>
\usepackage{fixltx2e} % LaTeX patches, \textsubscript
\usepackage{cmap} % fix search and cut-and-paste in Acrobat
\usepackage{ifthen}
\usepackage[T1]{fontenc}
\usepackage[utf8]{inputenc}

%%% Custom LaTeX preamble
% PDF Standard Fonts
\usepackage{mathptmx} % Times
\usepackage[scaled=.90]{helvet}
\usepackage{courier}

%%% User specified packages and stylesheets

%%% Fallback definitions for Docutils-specific commands
% numeric or symbol footnotes with hyperlinks
\providecommand*{\DUfootnotemark}[3]{%
  \raisebox{1em}{\hypertarget{#1}{}}%
  \hyperlink{#2}{\textsuperscript{#3}}%
}
\providecommand{\DUfootnotetext}[4]{%
  \begingroup%
  \renewcommand{\thefootnote}{%
    \protect\raisebox{1em}{\protect\hypertarget{#1}{}}%
    \protect\hyperlink{#2}{#3}}%
  \footnotetext{#4}%
  \endgroup%
}

% hyperlinks:
\ifthenelse{\isundefined{\hypersetup}}{
  \usepackage[colorlinks=true,linkcolor=blue,urlcolor=blue]{hyperref}
  \urlstyle{same} % normal text font (alternatives: tt, rm, sf)
}{}
\hypersetup{
  pdftitle={reStructuredTextで快適Web生活},
}

%%% Title Data
\title{\phantomsection%
  reStructuredTextで快適Web生活%
  \label{restructuredtextweb}}
\author{}
\date{}

%%% Body
\begin{document}
\maketitle


%___________________________________________________________________________

\section*{\phantomsection%
  reStructuredTextのメリット%
  \addcontentsline{toc}{section}{reStructuredTextのメリット}%
  \label{restructuredtext}%
}
%
\begin{itemize}

\item 手軽

\item wikiと異なり,データがローカルにある(個人で編集するならメリット)

\item 出力形式が多様

\end{itemize}


%___________________________________________________________________________

\section*{\phantomsection%
  gnupack環境をインストールと設定%
  \addcontentsline{toc}{section}{gnupack環境をインストールと設定}%
  \label{gnupack}%
}

簡単にcygwin環境とemacs等をインストールできるパッケージ,grupack\DUfootnotemark{id1}{f1}{1} をインストールします.
ここでは,ユーザのホームディレクトリ( C:\textbackslash{}Users\textbackslash{}<name>\textbackslash{} )\DUfootnotemark{id2}{f2}{2} にインストールします.
\newcounter{listcnt0}
\begin{list}{\arabic{listcnt0}.}
{
\usecounter{listcnt0}
\setlength{\rightmargin}{\leftmargin}
}

\item \href{http://sourceforge.jp/projects/gnupack/downloads/52709/gnupack_basic-7.00.exe/}{gnupack-basic-7.00.exe} をダウンロードします

\item ダブルクリックして実行します

\item ``解凍先''に C:\textbackslash{}Users\textbackslash{}<name>\textbackslash{} を指定します
\end{list}

新たに, C:\textbackslash{}Users\textbackslash{}<name>\textbackslash{}gnupack\_basic-7.00 ディレクトリができます.
その下にファイルが展開されていますので,確認してください.
%
\DUfootnotetext{f1}{id1}{1}{\phantomsection\label{f1}%
gnupackにはbasic版とdevelop版があります.
develop版のgccは今回利用しないので,basic版をインストールしてください.
}
%
\DUfootnotetext{f2}{id2}{2}{\phantomsection\label{f2}%
\textbackslash{} は「¥」記号に読み替えてください.
\emph{<name>} の部分は,あなたのログイン名に置き換えてください.
}


%___________________________________________________________________________

\section*{\phantomsection%
  cygwin用パッケージ管理ツール%
  \addcontentsline{toc}{section}{cygwin用パッケージ管理ツール}%
  \label{cygwin}%
}

cygwinにパッケージを追加するために, setup.exeをダウンロードします.
このツールを使って以下のパッケージを追加します.
%
\begin{itemize}

\item Python

\item curl

\item gcc

\end{itemize}


%___________________________________________________________________________

\subsection*{\phantomsection%
  準備%
  \addcontentsline{toc}{subsection}{準備}%
  \label{id4}%
}
\setcounter{listcnt0}{0}
\begin{list}{\arabic{listcnt0}.}
{
\usecounter{listcnt0}
\setlength{\rightmargin}{\leftmargin}
}

\item \href{http://cygwin.com/setup.exe}{setup.exe} をダウンロードします.

\item ダウンロードしたファイルを C:\textbackslash{}Users\textbackslash{}<name>\textbackslash{}gnupack\_basic-7.00 ディレクトリにコピーしてください.

\item また,同じ場所に``package''という名前でフォルダを作成してください.
\end{list}


%___________________________________________________________________________

\subsection*{\phantomsection%
  setup.exeの設定%
  \addcontentsline{toc}{subsection}{setup.exeの設定}%
  \label{id5}%
}
\setcounter{listcnt0}{0}
\begin{list}{\arabic{listcnt0}.}
{
\usecounter{listcnt0}
\setlength{\rightmargin}{\leftmargin}
}

\item setup.exeを実行します.

\item ``次へ''でページをめくり,``Cygwin Setup - Choose Installation Directory''画面に進みます.

\item ``Root Directory''に C:\textbackslash{}Users\textbackslash{}yc\textbackslash{}gnupack\_basic-7.00\textbackslash{}app\textbackslash{}cygwin\textbackslash{}cygwin を指定してください.

\item ``次へ''でページをめくり,``Cygwin Setup - Select Local Package Directory''画面に進みます.

\item ``Local Package Directory''に C:\textbackslash{}Users\textbackslash{}<name>\textbackslash{}gnupack\_basic-7.00\textbackslash{}package を指定してください.
\end{list}


%___________________________________________________________________________

\subsection*{\phantomsection%
  パッケージの追加%
  \addcontentsline{toc}{subsection}{パッケージの追加}%
  \label{id6}%
}
\setcounter{listcnt0}{0}
\begin{list}{\arabic{listcnt0}.}
{
\usecounter{listcnt0}
\setlength{\rightmargin}{\leftmargin}
}

\item ``次へ''でページをめくり,``Cygwin Setup - Select Packages''画面に進みます.

\item ``Search''に``python''と入力してください(画面が更新されるまでに若干ラグがあります).

\item ``All:Interpreters''の中にある``python: Python language interpreter''の行にある``Skip''をクリックします(表示がバージョン番号に変わります).

\item ``Search''に``curl''と入力してください.

\item ``All:Net''の中にある``curl: Multi-protocol file transfer command-line tool''の行にある``Skip''をクリックします(表示がバージョン番号に変わります).

\item ``Search''に``gcc''と入力してください.

\item ``Devel''の中にある``gcc: C compiler upgrade helper''の行にある``Skip''をクリックします(表示がバージョン番号に変わります).

\item ``次へ''でページをめくり,``Cygwin Setup''画面に進みます.

\item ダウンロードが終了すると,``Cygwin Setup - Installation Status and CReate Icons''画面がでます.

\item ``Crate icon on Desktop''のチェックは外して,``完了''を押してください.
\end{list}


%___________________________________________________________________________

\section*{\phantomsection%
  python用パッケージ管理ツール%
  \addcontentsline{toc}{section}{python用パッケージ管理ツール}%
  \label{python}%
}
\setcounter{listcnt0}{0}
\begin{list}{\arabic{listcnt0}.}
{
\usecounter{listcnt0}
\setlength{\rightmargin}{\leftmargin}
}

\item C:\textbackslash{}Users\textbackslash{}<name>\textbackslash{}gnupack\_basic-7.00\textbackslash{}mintty.exe を起動して,コマンドプロンプトを表示させます.

\item 次のコマンドを実行してください.
\end{list}
%
\begin{quote}{\ttfamily \raggedright \noindent
\#~curl~-O~http://peak.telecommunity.com/dist/ez\_setup.py\\
\#~python~ez\_setup.py
}
\end{quote}


%___________________________________________________________________________

\section*{\phantomsection%
  rst2pdfのインストール%
  \addcontentsline{toc}{section}{rst2pdfのインストール}%
  \label{rst2pdf}%
}


%___________________________________________________________________________

\subsection*{\phantomsection%
  cygwinパッケージの追加%
  \addcontentsline{toc}{subsection}{cygwinパッケージの追加}%
  \label{id7}%
}
%
\begin{itemize}

\item jpeg -> All:Graphics jpeg: A library for manipulating JPEG image format files

\item lcms -> All:Graphics lcms: Little color management engine - (Python bindings)

\item freetype -> freetype2: High-quality software font engine (sources)

\end{itemize}


%___________________________________________________________________________

\subsection*{\phantomsection%
  pythonパッケージの追加%
  \addcontentsline{toc}{subsection}{pythonパッケージの追加}%
  \label{id8}%
}
\setcounter{listcnt0}{0}
\begin{list}{\arabic{listcnt0}.}
{
\usecounter{listcnt0}
\setlength{\rightmargin}{\leftmargin}
}

\item C:\textbackslash{}Users\textbackslash{}<name>\textbackslash{}gnupack\_basic-7.00\textbackslash{}mintty.exe を起動して,コマンドプロンプトを表示させます.

\item 次のコマンドを実行してください.
\end{list}
%
\begin{quote}{\ttfamily \raggedright \noindent
\#~easy\_install~PIL\\
\#~easy\_install~reportlab\\
\#~easy\_install~rst2pdf
}
\end{quote}


%___________________________________________________________________________

\subsection*{\phantomsection%
  コンパイルエラーへの対処%
  \addcontentsline{toc}{subsection}{コンパイルエラーへの対処}%
  \label{id9}%
}

コンパイル中に以下のエラーになる場合があります.
%
\begin{quote}{\ttfamily \raggedright \noindent
***~fatal~error~-~unable~to~remap~\textbackslash{}\textbackslash{}?\textbackslash{}C:\textbackslash{}Users\textbackslash{}yc\textbackslash{}gnupack\_basic-7.00\textbackslash{}app\textbackslash{}cygwin\textbackslash{}cygwin\textbackslash{}lib\textbackslash{}python2.6\textbackslash{}lib-dynload\textbackslash{}time.dll~to~same~address~as~parent:~0x3D0000~!=~0x3F0000\\
Stack~trace:\\
Frame~~~~~Function~~Args\\
00224DF8~~6102796B~~(00224DF8,~00000000,~00000000,~00000000)\\
002250E8~~6102796B~~(6117EC60,~00008000,~00000000,~61180977)\\
00226118~~61004F1B~~(611A7FAC,~612483E4,~003D0000,~003F0000)\\
End~of~stack~trace\\
1~{[}main{]}~python~6956~fork:~child~7068~-~died~waiting~for~dll~loading,~errno~11\\
error:~Setup~script~exited~with~error:~Resource~temporarily~unavailable
}
\end{quote}

その場合, C:\textbackslash{}Users\textbackslash{}yc\textbackslash{}gnupack\_basic-7.00\textbackslash{}app\textbackslash{}cygwin\textbackslash{}cygwin の下にある\textbackslash{}bin\textbackslash{}ash.exeを起動して,次のコマンドを入力します.
%
\begin{quote}{\ttfamily \raggedright \noindent
\$~/bin/rebaseall\\
\$~exit
}
\end{quote}

(参考 \url{http://d.hatena.ne.jp/haradats/20080119/p4} )


%___________________________________________________________________________

\subsection*{\phantomsection%
  rest2pdf.pyの修正%
  \addcontentsline{toc}{subsection}{rest2pdf.pyの修正}%
  \label{rest2pdf-py}%
}

``/usr/lib/python2.6/site-packages/rst2pdf-0.16-py2.6.egg/rst2pdf/createpdf.py''

169行目
%
\begin{quote}{\ttfamily \raggedright \noindent
get\_language~(self.language,~None)
}
\end{quote}

を以下の通り変更.
%
\begin{quote}{\ttfamily \raggedright \noindent
get\_language~(self.language,~None)
}
\end{quote}

249行目
%
\begin{quote}{\ttfamily \raggedright \noindent
self.docutils\_languages{[}lang{]}~=~get\_language(lang)
}
\end{quote}

を
%
\begin{quote}{\ttfamily \raggedright \noindent
self.docutils\_languages{[}lang{]}~=~get\_language(lang,~None)
}
\end{quote}

(参考: \url{http://d.hatena.ne.jp/w650/20100216/1266287625} )


%___________________________________________________________________________

\subsection*{\phantomsection%
  reportlabの修正%
  \addcontentsline{toc}{subsection}{reportlabの修正}%
  \label{reportlab}%
}
%
\begin{itemize}

\item \url{http://harajuku-tech.posterous.com/rst2pdfpdf}

\end{itemize}


%___________________________________________________________________________

\subsection*{\phantomsection%
  フォントの追加%
  \addcontentsline{toc}{subsection}{フォントの追加}%
  \label{id10}%
}
\setcounter{listcnt0}{0}
\begin{list}{\arabic{listcnt0}.}
{
\usecounter{listcnt0}
\setlength{\rightmargin}{\leftmargin}
}

\item \textasciitilde{}/fontを作成

\item IPAフォントをダウンロードする
\end{list}


%___________________________________________________________________________

\subsection*{\phantomsection%
  PDFの生成%
  \addcontentsline{toc}{subsection}{PDFの生成}%
  \label{pdf}%
}
\setcounter{listcnt0}{0}
\begin{list}{\arabic{listcnt0}.}
{
\usecounter{listcnt0}
\setlength{\rightmargin}{\leftmargin}
}

\item ja.jsonを作成

\item コマンド実行
\end{list}


%___________________________________________________________________________

\section*{\phantomsection%
  参考Webページ%
  \addcontentsline{toc}{section}{参考Webページ}%
  \label{web}%
}
%
\begin{itemize}

\item \url{http://d.hatena.ne.jp/shinriyo/20110810} -> 英語が通るように

\item \url{http://d.hatena.ne.jp/hokorobi/20101031/1288513452}

\item \url{http://resteditor.sourceforge.net/}

\end{itemize}

\end{document}
