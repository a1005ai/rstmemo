% Generated by Sphinx.
\def\sphinxdocclass{report}
\documentclass[letterpaper,10pt,english]{sphinxmanual}
\usepackage[utf8]{inputenc}
\DeclareUnicodeCharacter{00A0}{\nobreakspace}
\usepackage[T1]{fontenc}
\usepackage{babel}
\usepackage{times}
\usepackage[Bjarne]{fncychap}
\usepackage{longtable}
\usepackage{sphinx}


\title{Sphinx memo Documentation}
\date{August 08, 2011}
\release{0.1}
\author{Y.Chubachi}
\newcommand{\sphinxlogo}{}
\renewcommand{\releasename}{Release}
\makeindex

\makeatletter
\def\PYG@reset{\let\PYG@it=\relax \let\PYG@bf=\relax%
    \let\PYG@ul=\relax \let\PYG@tc=\relax%
    \let\PYG@bc=\relax \let\PYG@ff=\relax}
\def\PYG@tok#1{\csname PYG@tok@#1\endcsname}
\def\PYG@toks#1+{\ifx\relax#1\empty\else%
    \PYG@tok{#1}\expandafter\PYG@toks\fi}
\def\PYG@do#1{\PYG@bc{\PYG@tc{\PYG@ul{%
    \PYG@it{\PYG@bf{\PYG@ff{#1}}}}}}}
\def\PYG#1#2{\PYG@reset\PYG@toks#1+\relax+\PYG@do{#2}}

\def\PYG@tok@gd{\def\PYG@tc##1{\textcolor[rgb]{0.63,0.00,0.00}{##1}}}
\def\PYG@tok@gu{\let\PYG@bf=\textbf\def\PYG@tc##1{\textcolor[rgb]{0.50,0.00,0.50}{##1}}}
\def\PYG@tok@gt{\def\PYG@tc##1{\textcolor[rgb]{0.00,0.25,0.82}{##1}}}
\def\PYG@tok@gs{\let\PYG@bf=\textbf}
\def\PYG@tok@gr{\def\PYG@tc##1{\textcolor[rgb]{1.00,0.00,0.00}{##1}}}
\def\PYG@tok@cm{\let\PYG@it=\textit\def\PYG@tc##1{\textcolor[rgb]{0.25,0.50,0.56}{##1}}}
\def\PYG@tok@vg{\def\PYG@tc##1{\textcolor[rgb]{0.73,0.38,0.84}{##1}}}
\def\PYG@tok@m{\def\PYG@tc##1{\textcolor[rgb]{0.13,0.50,0.31}{##1}}}
\def\PYG@tok@mh{\def\PYG@tc##1{\textcolor[rgb]{0.13,0.50,0.31}{##1}}}
\def\PYG@tok@cs{\def\PYG@tc##1{\textcolor[rgb]{0.25,0.50,0.56}{##1}}\def\PYG@bc##1{\colorbox[rgb]{1.00,0.94,0.94}{##1}}}
\def\PYG@tok@ge{\let\PYG@it=\textit}
\def\PYG@tok@vc{\def\PYG@tc##1{\textcolor[rgb]{0.73,0.38,0.84}{##1}}}
\def\PYG@tok@il{\def\PYG@tc##1{\textcolor[rgb]{0.13,0.50,0.31}{##1}}}
\def\PYG@tok@go{\def\PYG@tc##1{\textcolor[rgb]{0.19,0.19,0.19}{##1}}}
\def\PYG@tok@cp{\def\PYG@tc##1{\textcolor[rgb]{0.00,0.44,0.13}{##1}}}
\def\PYG@tok@gi{\def\PYG@tc##1{\textcolor[rgb]{0.00,0.63,0.00}{##1}}}
\def\PYG@tok@gh{\let\PYG@bf=\textbf\def\PYG@tc##1{\textcolor[rgb]{0.00,0.00,0.50}{##1}}}
\def\PYG@tok@ni{\let\PYG@bf=\textbf\def\PYG@tc##1{\textcolor[rgb]{0.84,0.33,0.22}{##1}}}
\def\PYG@tok@nl{\let\PYG@bf=\textbf\def\PYG@tc##1{\textcolor[rgb]{0.00,0.13,0.44}{##1}}}
\def\PYG@tok@nn{\let\PYG@bf=\textbf\def\PYG@tc##1{\textcolor[rgb]{0.05,0.52,0.71}{##1}}}
\def\PYG@tok@no{\def\PYG@tc##1{\textcolor[rgb]{0.38,0.68,0.84}{##1}}}
\def\PYG@tok@na{\def\PYG@tc##1{\textcolor[rgb]{0.25,0.44,0.63}{##1}}}
\def\PYG@tok@nb{\def\PYG@tc##1{\textcolor[rgb]{0.00,0.44,0.13}{##1}}}
\def\PYG@tok@nc{\let\PYG@bf=\textbf\def\PYG@tc##1{\textcolor[rgb]{0.05,0.52,0.71}{##1}}}
\def\PYG@tok@nd{\let\PYG@bf=\textbf\def\PYG@tc##1{\textcolor[rgb]{0.33,0.33,0.33}{##1}}}
\def\PYG@tok@ne{\def\PYG@tc##1{\textcolor[rgb]{0.00,0.44,0.13}{##1}}}
\def\PYG@tok@nf{\def\PYG@tc##1{\textcolor[rgb]{0.02,0.16,0.49}{##1}}}
\def\PYG@tok@si{\let\PYG@it=\textit\def\PYG@tc##1{\textcolor[rgb]{0.44,0.63,0.82}{##1}}}
\def\PYG@tok@s2{\def\PYG@tc##1{\textcolor[rgb]{0.25,0.44,0.63}{##1}}}
\def\PYG@tok@vi{\def\PYG@tc##1{\textcolor[rgb]{0.73,0.38,0.84}{##1}}}
\def\PYG@tok@nt{\let\PYG@bf=\textbf\def\PYG@tc##1{\textcolor[rgb]{0.02,0.16,0.45}{##1}}}
\def\PYG@tok@nv{\def\PYG@tc##1{\textcolor[rgb]{0.73,0.38,0.84}{##1}}}
\def\PYG@tok@s1{\def\PYG@tc##1{\textcolor[rgb]{0.25,0.44,0.63}{##1}}}
\def\PYG@tok@gp{\let\PYG@bf=\textbf\def\PYG@tc##1{\textcolor[rgb]{0.78,0.36,0.04}{##1}}}
\def\PYG@tok@sh{\def\PYG@tc##1{\textcolor[rgb]{0.25,0.44,0.63}{##1}}}
\def\PYG@tok@ow{\let\PYG@bf=\textbf\def\PYG@tc##1{\textcolor[rgb]{0.00,0.44,0.13}{##1}}}
\def\PYG@tok@sx{\def\PYG@tc##1{\textcolor[rgb]{0.78,0.36,0.04}{##1}}}
\def\PYG@tok@bp{\def\PYG@tc##1{\textcolor[rgb]{0.00,0.44,0.13}{##1}}}
\def\PYG@tok@c1{\let\PYG@it=\textit\def\PYG@tc##1{\textcolor[rgb]{0.25,0.50,0.56}{##1}}}
\def\PYG@tok@kc{\let\PYG@bf=\textbf\def\PYG@tc##1{\textcolor[rgb]{0.00,0.44,0.13}{##1}}}
\def\PYG@tok@c{\let\PYG@it=\textit\def\PYG@tc##1{\textcolor[rgb]{0.25,0.50,0.56}{##1}}}
\def\PYG@tok@mf{\def\PYG@tc##1{\textcolor[rgb]{0.13,0.50,0.31}{##1}}}
\def\PYG@tok@err{\def\PYG@bc##1{\fcolorbox[rgb]{1.00,0.00,0.00}{1,1,1}{##1}}}
\def\PYG@tok@kd{\let\PYG@bf=\textbf\def\PYG@tc##1{\textcolor[rgb]{0.00,0.44,0.13}{##1}}}
\def\PYG@tok@ss{\def\PYG@tc##1{\textcolor[rgb]{0.32,0.47,0.09}{##1}}}
\def\PYG@tok@sr{\def\PYG@tc##1{\textcolor[rgb]{0.14,0.33,0.53}{##1}}}
\def\PYG@tok@mo{\def\PYG@tc##1{\textcolor[rgb]{0.13,0.50,0.31}{##1}}}
\def\PYG@tok@mi{\def\PYG@tc##1{\textcolor[rgb]{0.13,0.50,0.31}{##1}}}
\def\PYG@tok@kn{\let\PYG@bf=\textbf\def\PYG@tc##1{\textcolor[rgb]{0.00,0.44,0.13}{##1}}}
\def\PYG@tok@o{\def\PYG@tc##1{\textcolor[rgb]{0.40,0.40,0.40}{##1}}}
\def\PYG@tok@kr{\let\PYG@bf=\textbf\def\PYG@tc##1{\textcolor[rgb]{0.00,0.44,0.13}{##1}}}
\def\PYG@tok@s{\def\PYG@tc##1{\textcolor[rgb]{0.25,0.44,0.63}{##1}}}
\def\PYG@tok@kp{\def\PYG@tc##1{\textcolor[rgb]{0.00,0.44,0.13}{##1}}}
\def\PYG@tok@w{\def\PYG@tc##1{\textcolor[rgb]{0.73,0.73,0.73}{##1}}}
\def\PYG@tok@kt{\def\PYG@tc##1{\textcolor[rgb]{0.56,0.13,0.00}{##1}}}
\def\PYG@tok@sc{\def\PYG@tc##1{\textcolor[rgb]{0.25,0.44,0.63}{##1}}}
\def\PYG@tok@sb{\def\PYG@tc##1{\textcolor[rgb]{0.25,0.44,0.63}{##1}}}
\def\PYG@tok@k{\let\PYG@bf=\textbf\def\PYG@tc##1{\textcolor[rgb]{0.00,0.44,0.13}{##1}}}
\def\PYG@tok@se{\let\PYG@bf=\textbf\def\PYG@tc##1{\textcolor[rgb]{0.25,0.44,0.63}{##1}}}
\def\PYG@tok@sd{\let\PYG@it=\textit\def\PYG@tc##1{\textcolor[rgb]{0.25,0.44,0.63}{##1}}}

\def\PYGZbs{\char`\\}
\def\PYGZus{\char`\_}
\def\PYGZob{\char`\{}
\def\PYGZcb{\char`\}}
\def\PYGZca{\char`\^}
\def\PYGZsh{\char`\#}
\def\PYGZpc{\char`\%}
\def\PYGZdl{\char`\$}
\def\PYGZti{\char`\~}
% for compatibility with earlier versions
\def\PYGZat{@}
\def\PYGZlb{[}
\def\PYGZrb{]}
\makeatother

\begin{document}

\maketitle
\tableofcontents
\phantomsection\label{index::doc}


Contents:


\chapter{reStructuredTextで快適Web生活}
\label{test::doc}\label{test:restructuredtextweb}

\section{ToDo}
\label{test:todo}\begin{enumerate}
\item {} 
latex処理をできるようにする

\item {} 
テンプレートをカスタマイズしてみる

\item {} 
rst2pdfのインストール

\item {} 
emacsからmakeできるようにする

\end{enumerate}


\section{このページへのURL}
\label{test:url}
\href{http://dl.dropbox.com/u/1312957/sphinx/rstmemo/\_build/html/index.html}{here}


\section{gnupack環境}
\label{test:gnupack}\begin{enumerate}
\item {} 
gnupack \href{http://gnupack.sourceforge.jp/docs/current/UsersGuide.html}{http://gnupack.sourceforge.jp/docs/current/UsersGuide.html}

\end{enumerate}

gnupackはアーカイブ展開のみで即座に cygwinと emacsが使える環境を提供します.


\section{Sphinxのインストール}
\label{test:sphinx}

\subsection{Dropboxにフォルダを作り, リンクを張る}
\label{test:dropbox}
\begin{Verbatim}[commandchars=\\\{\}]
\PYG{c}{\PYGZsh{} mkdir /c/Users/yc/My\PYGZbs{} Documents/My\PYGZbs{} Dropbox/Public public}
\PYG{c}{\PYGZsh{} ln -s  /c/Users/yc/My\PYGZbs{} Documents/My\PYGZbs{} Dropbox/Public public}
\end{Verbatim}


\subsection{Pythonのインストール}
\label{test:python}
\begin{Verbatim}[commandchars=\\\{\}]
\PYG{c}{\PYGZsh{} apt-cyg install python}
\end{Verbatim}


\subsection{curlのインストール}
\label{test:curl}
\begin{Verbatim}[commandchars=\\\{\}]
\PYG{c}{\PYGZsh{} apt-cyg install curl}
\end{Verbatim}


\subsection{easy\_installのインストール}
\label{test:easy-install}
\begin{Verbatim}[commandchars=\\\{\}]
\PYG{c}{\PYGZsh{} curl -O http://peak.telecommunity.com/dist/ez\PYGZus{}setup.py}
\PYG{c}{\PYGZsh{} ./ez\PYGZus{}setup.py}
\end{Verbatim}


\subsection{Sphinxのインストール}
\label{test:id1}
\href{http://sphinx-users.jp/gettingstarted/install\_windows.html\#sphinx}{http://sphinx-users.jp/gettingstarted/install\_windows.html\#sphinx}

\begin{Verbatim}[commandchars=\\\{\}]
\PYG{c}{\PYGZsh{} easy\PYGZus{}install sphinx}
\end{Verbatim}


\section{reStructuredTextの情報}
\label{test:restructuredtext}\begin{itemize}
\item {} 
\href{http://sphinx-users.jp/doc10/}{Sphinx 1.0}

\item {} 
\href{http://sphinx-users.jp/doc.html}{Sphinx-Users.jp}

\item {} 
\href{http://sphinx.shibu.jp/rest.html}{reStructuredText入門}

\item {} 
\href{http://lateral.netmanagers.com.ar/stories/BBS52.html}{Creating presentations using restructured text}

\item {} 
\href{http://docutils.sourceforge.net/docs/user/rst/quickref.html}{User refference}

\end{itemize}


\section{LaTeX}
\label{test:latex}\begin{itemize}
\item {} 
pLaTexを別途インストールする

\item {} 
pathを通す

\item {} 
パッケージのインストール \href{http://oku.edu.mie-u.ac.jp/~okumura/texwiki/?TeX\%E5\%85\%A5\%E9\%96\%80\%2F\%E5\%90\%84\%E7\%A8\%AE\%E3\%83\%91\%E3\%83\%83\%E3\%82\%B1\%E3\%83\%BC\%E3\%82\%B8\%E3\%81\%AE\%E5\%88\%A9\%E7\%94\%A8}{http://oku.edu.mie-u.ac.jp/\textasciitilde{}okumura/texwiki/?TeX\%E5\%85\%A5\%E9\%96\%80\%2F\%E5\%90\%84\%E7\%A8\%AE\%E3\%83\%91\%E3\%83\%83\%E3\%82\%B1\%E3\%83\%BC\%E3\%82\%B8\%E3\%81\%AE\%E5\%88\%A9\%E7\%94\%A8}

\end{itemize}

\begin{Verbatim}[commandchars=\\\{\}]
\PYG{c}{\PYGZsh{} The language for content autogenerated by Sphinx. Refer to documentation}
\PYG{c}{\PYGZsh{} for a list of supported languages.}
\PYG{n}{language} \PYG{o}{=} \PYG{l+s}{'}\PYG{l+s}{ja}\PYG{l+s}{'}
\end{Verbatim}


\chapter{メリット}
\label{test:id3}\begin{itemize}
\item {} 
手軽

\item {} 
wikiと異なり,データがローカルにある(個人で編集するならメリット)

\item {} 
出力形式が多様

\end{itemize}


\chapter{ePub}
\label{epub::doc}\label{epub:epub}

\section{WYSIWYG}
\label{epub:wysiwyg}\begin{itemize}
\item {} 
\href{http://code.google.com/p/sigil/}{EPUB出力ができるWYSIWYGエディタ「sigil」}

\end{itemize}

oeusnhsnh
sntouahsn


\chapter{Indices and tables}
\label{index:indices-and-tables}\begin{itemize}
\item {} 
\emph{genindex}

\item {} 
\emph{modindex}

\item {} 
\emph{search}

\end{itemize}



\renewcommand{\indexname}{Index}
\printindex
\end{document}
